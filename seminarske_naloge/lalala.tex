\documentclass[11pt,a4paper]{article}

\usepackage[utf8]{inputenc}  % omogoča uporabo slovenskih črk kodiranih v formatu UTF-8
\usepackage[slovene]{babel}    % naloži, med drugim, slovenske delilne vzorce

\usepackage[
backend=biber,
style=numeric,
sorting=none,
]{biblatex}

\usepackage{hyperref}


\title{Motion capture for Blender}
\author{Gašper Krevs\\
gasper.krevs@gmail.com\\
\ \\
možni MENTOR: (viš. pred./doc./prof.) dr. Borut Batagelj \\
Fakulteta za računalništvo in informatiko\\
Univerza v Ljubljani
\date{\today}         
}

\addbibresource{literatura.bib} %Imports bibliography file

\begin{document}
\maketitle

Napišite dokument na dveh do treh straneh o svoji načrtovani diplomski temi.
Za generiranje seznama literature uporabite BibLaTeX. 

V seznamu literature in zato tudi v besedilu ustrezni sklici na to literaturo, na primer \cite{ravnik2013audience}, naj bo:
\begin{itemize}
\item 
vsaj ena diploma FRI, ki je po tematiki čimbolj sorodna vaši diplomski temi,
\item
v seznamu literature naj bosta vsaj dva članka iz znanstvenih revij (tip literature v BibLaTeXu: article), 
\item
vsaj pri enem članku naj bo eden od avtorjev pedagog na FRI,
\item
spletna stran.
\end{itemize}
V datoteki \textit{literatura.bib} so primeri različnih vrst referenc!



\section{Motiv za diplomsko nalogo}

Zakaj me ta tema zanima? 
Komu in kako bodo rezultati diplomske naloge koristili?

\section{Ali je že bila kakšna diplomska ali magistrska naloga na podobno temo?}

Preišči repozitorij FRI \url{http://eprints.fri.uni-lj.si}
in UL \url{https://repozitorij.uni-lj.si}.


\section{Ali se je s to tematiko že ukvarjal kakšen učitelj na FRI?}

Povzemi glavne rezultate teh raziskav in navedi ustrezne članke v seznamu literature!

Pomagaj si z iskanjem po \url{https://scholar.google.si} in drugih akademskih spletnih portalih.


\section{Kaj je konkretni cilj diplomske naloge in kateri so glavni koraki do tega cilja?}

Naštej po točkah bistvene korake do uresničitve cilja.


\section{S kakšnimi orodji boš prišel do cilja?}

Kakšne metode, strojno opremo in druga orodja boš uporabljali pri izdelavi diplomske naloge?


\section{Kako boš preizkusil rešitev ali ustreza zadanim ciljem?}

Ali obstajajo kakšne podatkovne zbirke namenjene testiranju rezultatov?


\section{Zaključek: zakaj je izbrani oz. željeni mentor primeren za predlagano temo?}

Avtor člankov? Nosilec predmeta? Vodja laboratorija?

\printbibliography

\end{document}  